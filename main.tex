\documentclass[a4paper,10pt]{article}
\usepackage[left=0.2in, right=0.2in, top=0.2in, bottom=0.2in]{geometry}
\usepackage{enumitem}
\usepackage{hyperref}
\usepackage{titlesec}
\usepackage{multicol}
\usepackage{parskip}
\usepackage{titling}
\usepackage{graphicx}
\usepackage{color}
\usepackage{ulem}
\usepackage{fontspec}
\setmainfont{Tinos.ttf}[
    Path = ./,
    Extension = .ttf,
    UprightFont = *-Regular,
    BoldFont = *-Bold,
    ItalicFont = *-Italic,
    BoldItalicFont = *-BoldItalic
]

\usepackage{xcolor}
\definecolor{skyblue}{RGB}{0, 153, 255}
\hypersetup{
    colorlinks=true,     
    urlcolor=skyblue,
}

\hyphenpenalty=10000
\exhyphenpenalty=10000
\sloppy


\newcommand{\sectionspacing}{\vspace{-0.2em}}

% Section formatting with centered headings and lines
\titleformat{\section}
  {\fontsize{13}{15}\selectfont\bfseries\centering} % Section font size and style
  {}
  {0em}
  {}

\titlespacing*{\section}{0pt}{1em}{0.5em}

% Remove indentation
\setlength{\parindent}{0pt}

\begin{document}

% Header
\begin{center}
    {\fontsize{22}{24}\selectfont\textbf{Subhrasmit Kar}} \par
    
    \vspace{-0.3em}
    
    Chinsurah, Hooghly, West Bengal, India \par
    
    \vspace{-0.3em}
    
    \href{tel:+916289054902}{+91 6289054902} \quad | \quad \href{mailto:karsubhrasmit@gmail.com}{karsubhrasmit@gmail.com} \par
    
    \vspace{-0.5em}
    
    \href{https://github.com/subhrasmit17}{github.com/subhrasmit17} \quad | \quad \href{https://linkedin.com/in/subhrasmit}{linkedin.com/in/subhrasmit}
\end{center}



\vspace{-0.3em} %to preverve the distance b/w subhrasmit Kar and summary while changing space b/w other sub-headings



% Summary
\section*{Summary}
Computer Science engineer skilled in C, Java, SQL, and full-stack web development. Experienced in cybersecurity and encryption model development, with hands-on expertise in deploying web applications using Spring Boot, Maven, Docker, and Render. Equipped with robust coding and problem-solving abilities, reinforced by real-world projects and continuous self-learning. Passionate about leading teams and solving complex technical challenges.






% Technical Skills
\section*{Technical Skills}
\textbf{Programming Languages:} C, Java, SQL, C++, HTML, CSS, JavaScript, LaTeX \\[0.5em]
\textbf{Tools \& Technologies:} Git, VS Code, MySQL, IntelliJ IDEA, Postman, Spring Boot, Maven, Docker, Overleaf, OOP, DSA \\[0.5em]
\textbf{Cybersecurity \& Cryptography:} Image Encryption Methodology, Bit-Plane Slicing, DNA Coding, Chaos Theory \\[0.5em]
\textbf{Soft Skills:} Leadership, Problem-solving, Time Management, Communication, Adaptability, Fast Learner, Ethics






% Education
\section*{Education}
\textbf{Bachelor of Technology (2021 - 2025)} \hfill {\textbf{CGPA: 7.95}} \\
Computer Science \& Engineering \hfill Hooghly Engineering and Technology College %\\
%\makebox[\linewidth][r]{MAKAUT University}

\vspace{0.5em}

\textbf{ISC (2021)} \hfill {\textbf{Percentage: 90.25\%}} \\
Physics, Chemistry, Maths, Computer Science \hfill Don Bosco School Bandel

\vspace{0.5em}

\textbf{ICSE (2019)} \hfill {\textbf{Percentage: 93.00\%}} \\
Secondary Education \hfill Don Bosco School Bandel






% Projects
\section*{Projects}
\textbf{Image Encryption using Bit-Plane Slicing \& DNA Encoding} \hfill {\textbf{September 2024 - May 2025}}\\[0.5em]
\textbf{Technologies Used:} Java, Cryptography, Bit-Plane Slicing, DNA Encoding, Chaos Theory\hfill {\href{https://github.com/subhrasmit17/Image-Encryption-Offline}{GitHub}}

\begin{itemize}[leftmargin=*]
    \item Led the design and implementation of a symmetric image encryption/decryption scheme ensuring secure and reversible image transformation.
    \item Designed a multistage encryption pipeline leveraging chaotic maps, DNA-based coding, and bit-plane slicing to maximize confusion and diffusion in image data.
    \item Engineered a symmetric decryption process to precisely reverse each encryption step, ensuring lossless recovery of the original image.
    \item Achieved strong security performance with NPCR (99.62\%), UACI (34.58\%), and favorable PSNR, validating the scheme's robustness for secure image storage and transmission.
\end{itemize}


\vspace{0.5em}


\textbf{Image Encryption Web Application} \hfill {\textbf{June 2025 - Present}}\\[0.5em]
\textbf{Technologies Used:} Spring Boot, Maven, Docker, Render, HTML, CSS, JavaScript \hfill {\href{https://github.com/subhrasmit17/Image-Encryption-Online}{GitHub}}\\[0.5em]
Live Website: \href{https://image-encryptor-online.onrender.com}{https://image-encryptor-online.onrender.com}

\begin{itemize}[leftmargin=*]
    \item Independently built a live web application implementing the image encryption/decryption algorithm from my previous project, enabling real-time secure image processing via a web interface.
    \item Built a Spring Boot backend for efficient image handling, integrated with a responsive front-end using HTML, CSS, and JavaScript for a seamless user experience.
    \item Deployed the containerized application on Render using Docker, ensuring global accessibility with minimal cold start delays and optimized application responsiveness.
    \item Achieved continuous deployment via GitHub-integrated auto-deploy, enabling automated updates to the web application on every code push.
\end{itemize}






% Internships/Training
\section*{Internships / Training}
\vspace{0.1em}
\textbf{Rifle Factory Ishapore, Advanced Weapon and Equipment India Limited} \hfill {\textbf{January 2025}}\\[-1.4em]

\begin{itemize}[leftmargin=*]
    \item Completed a two-week cybersecurity training on vulnerability identification, network security, and cryptography, learning how to strengthen an organization’s security frameworks.
    \item Acquired practical knowledge of Information Security Protocols, Human Resource Security, Asset Management, Physical Security, and Incident Response.
\end{itemize}






% Certifications
% \section*{Certifications}
% \begin{itemize}[leftmargin=*]
%     \item The Joy of Computing Using Python \textbf{(NPTEL)}
%     \item Problem Solving Through Programming in C \textbf{(NPTEL)}
%     \item Database Management System \textbf{(Great Learning Academy)}
%     \item Programming for Everybody (Getting Started With Python) \textbf{(Coursera)}
% \end{itemize}






% Languages
\section*{Languages}

\begin{center}
    \begin{tabular}{c @{\hspace{6em}} c @{\hspace{6em}} c} % Use 'c' for centering each column
        English (Fluent) & Hindi (Proficient) & Bengali (Native)
    \end{tabular}
\end{center}

\end{document}